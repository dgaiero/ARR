\documentclass[10pt]{article}

\usepackage{geometry}
\usepackage{titlesec}
\geometry{letterpaper,tmargin=1in,bmargin=1in,lmargin=1.4in,rmargin=1.4in}

\titleformat{\section}
{\normalfont\Large\bfseries}{\S\thesection}{1em}{}

\begin{document}
	\begin{center}
		{\Large Apartment Rules and Regulations}
	\end{center}
\begin{center}
	
	{\normalsize  A.R.R.}
\end{center}
\hrule height .1mm

\vspace{.5cm}

\noindent This Agreement is dated and in effect as of the [Nth of MONTH, YEAR], between:\\
\\
{\bf PERSON 1}\\
{\bf PERSON 2}\\
{\bf PERSON 3}\\
{\bf PERSON 4}\\
{\bf PERSON 5}\\
{\bf PERSON 6}\\
\\(hereinafter the ``Client'', ``tenant'', or ``any party'')\\ \\

This contract governs how the contracting parties behave in this apartment with mailing address 055 Cerro Vista Circle, San Luis Obispo California, 93410-1702 (henceforth called ‘this apartment’ or ‘the apartment’). This contract shall become effective (the “Effective Date”) upon the date all of the contracting parties (hence forth called ‘the tenant’ or ‘any party’) have signed this document. This document shall stay in effect until the twenty-fifth of March at 11:59:59 PM, at which point this Contract will terminate.
\\ \\
\noindent REPRESENTATION ON AUTHORITY OF PARTIES/SIGNATORIES: EACH PER-SON SIGNING THIS CONTRACT REPRESENTS AND WARRANTS THAT HE OR SHE IS DULY AUTHORIZED AND HAS LEGAL CAPACITY TO EXECUTE AND DELIVER THIS CONTRACT. EACH PARTY REPRESENTS AND WARRANTS TO THE OTHER THAT THE EXECUTION AND DELIVERY OF THE CONTRACT AND THE PERFORMANCE OF SUCH PARTY’S OBLIGATIONS HEREUNDER HAVE BEEN DULY AUTHORIZED AND THAT THE CONTRACT IS A VALID AND LEGAL AGREEMENT BINDING ON SUCH PARTY AND ENFORCEABLE IN ACCORDANCE WITH ITS TERMS. SEE THE FOLLOWING PAGE FOR SIGNATURES.

\newpage
\section{Footwear}
This section covers the protocol with respect to footwear in and around the apartment.
\begin{enumerate}
	\item Once entering the apartment, all footwear will be removed.
	\item Footwear can either be placed in the designated footwear storage bin or in one’s University ap-pointed bedroom.
	\item Footwear that is permitted on premises is limited to: Sandals that are worn indoor only, or in exterior spaces where copious amounts of dirt are not furnished.
	\item If entering the apartment for less than five minutes, the footwear is permitted to stay on one’s feet
	\item This rule will be enforced upon all guests and visitors who enter this apartment
	
\end{enumerate}

\subsection{Footwear - more information}
\begin{enumerate}
	\item Footwear includes, but is not limited to shoes, boots, sandals, or any other item that is worn over the foot. Footwear does not include socks, slippers, and indoor sandals. Use discretion.
	\item Footwear that enters the apartment will not be covered in any amount of dirt and/or mud. Should footwear have any mud or water, that footwear is not permitted to touch the floor of the apartment. This footwear will be removed outside of the apartment.
	
\end{enumerate}

\section{Kitchen}
This section covers the protocol with respect to cleaning and maintain the apartment kitchen.
\begin{enumerate}
	\item Once a dish is used, it may be kept in the sink for no longer than one hour.
	\item If the preparation of the dish requires use of any other tools, each tool will also be cleaned and placed in the drying rack no more than one hour after the previously stated tool is used. 
	\item Should any part of the countertop, stovetop, or table (either kitchen table or coffee table) be used to place any dish or tool used, it must be cleaned no more than one hour after use.
	\item If any food should spill (either on the ground, or any other surface) it will be wiped up immediately.
	\item Should Any dish be broken, it will be replaced within a timely manner.
	When a roll of paper towel is completely used, 
	
\end{enumerate}
\subsection{Kitchen - more information}
\begin{enumerate}
	\item Exceptions to keeping this dish in the sink for longer than one hour: Should you use a dish and be required to leave prematurely, the dish may be cleaned at the end of the day (End of day is 11:59:59 PM).
	\item Tools include, but are not limited to: forks, knives, spoons, cutting boards, or any other item found in this apartment or borrowed items from other apartments.
	\item Timely manner is defined as: twenty-eight days.
	
\end{enumerate}

\section{Cleanliness}
Clean is defined as all personal property each tenant is in that tenant’s university assigned room, trash is removed, floors are swept, counters and bathrooms are cleaned off and no junk or trash is visible except for in the designated trash receptacles. No food or food remnants will be left in the apartment, the refrigerator and stove will "wiped down" and there will be no crusted over food or spills. The bathrooms will be in a clean state and free of excrement and urine in the toilets.
\begin{enumerate}
	\item No personal artifacts will be kept in the common areas.
	\item No food products or food wrappers will be left in any common area or floor of the study or in a bathroom.
	\item All jackets will be placed on their designated hooks.
	\item No backpack will be placed in any common areas. Backpacks will reside in the study or in the tenant’s university assigned room.
	\item Any papers will not be placed in any common areas. Any paper placed in these areas are subject to be thrown away.
	\item Any other items left unattended in these areas that are not permanent fixtures of the apartment are subject to removal.
	
\end{enumerate}
\subsection{Cleanliness - more information}
\begin{enumerate}
	\item Common area defined as: entry from door, kitchen, living room, both hallways. Any area that is not a university assigned bedroom or the study is considered a common area.
	\item A permanent fixture of the apartment is defined as: Any object that was originally in the apartment on move-in day, or a product that was purchased for the entire apartment to use (i.e. the television, etc.)
\end{enumerate}
\section{Sound}
\begin{enumerate}
	\item No excessive sounds are permitted any time of the day. As the time of day changes, what is considered excessive changes. For instance, it is acceptable to play the stereo at 12:00:00 PM, but not in the night when one of the tenants are sleeping.
	\item Screaming is prohibited any time of the day, except during emergencies.
	\item Once any contracting party sleeps, the rest of the contracting parties, guests, and visitors will reduce their voices, music, or any device that makes sound to no more than 82dB. 82dB is equivalent to the sound that a garbage disposal makes.
	\item No object will be thrown against any wall when a tenant is sleeping.
	\item The doorbell may not be rung between the hours of 11:30:00 PM to 08:00:00 AM unless the person ringing the doorbell does not have a means of entry into the apartment.
	
\end{enumerate}
\subsection{Sound - more information}
\begin{enumerate}
	\item Sleeping is defined as once any tenant enters their university assigned bedroom at night with the intent to sleep.
\end{enumerate}



\end{document}